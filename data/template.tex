\documentclass{article}
\usepackage{pgfplots}
\usepackage[a4paper, margin=1in]{geometry} % Adjust the margins here
\begin{document}

\title{Programação Concorrente e Distribuída}
\author{        }
\date{\today}

\maketitle


\section{Especificação} \label{sec:especificacao}
\begin{itemize}
    \item Modelo CPU: ${MODEL}
    \item Modelo GPU: ${GPU_MODEL}
    \item Total de nucleos: ${CPU}
    \item Total de nucleos fisicos: ${CPU_PHYSICAL}
    \item Mémoria RAM total: ${MEMORY}
\end{itemize}


\section{Tempo de execução: Table} \label{sec:tempo_de_execucao}
\begin{table}[h!]
    \centering
    \begin{tabular}{|c|c|}
        
    \hline
    \textbf{Número de processos} & \textbf{Tempo de execução (s)} \\
    \hline
    ${TABLE}
    \hline
    \end{tabular}
    \caption{Ultimo resultado obtido com CUDA}
    \end{table}

\section{Tempo de execução: Plot} \label{sec:tempo_de_execucao}
\begin{center} % Centering the plot
    \begin{tikzpicture}
        \begin{axis}[
            width=15cm, height=8cm, % Increase width
            xlabel={Tempo de execução (s)},
            ylabel={Número de processos},
            xtick=data,
            ytick=data, 
            grid=major,
            legend pos=north west
        ]
            % Line graph
            \addplot[color=blue, mark=none] coordinates {
                ${PLOT}
            };
            \addlegendentry{Line}

            % Point graph
            \addplot[color=red, only marks, mark=*] coordinates {
                  ${PLOT}
            };
            \addlegendentry{Points}
        \end{axis}
    \end{tikzpicture}
\end{center}
        
        \end{document}
